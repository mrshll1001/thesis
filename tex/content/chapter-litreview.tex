\chapter{Background and Literature}
\label{sec:related}

This chapter discusses the background and literature of the various fields that inform this research. Given the interdisciplinary history of HCI research, and the socio-economic domain of non-profit enterprise, it should come of no surprise that this results in a rich and diverse nexus of perspectives which needs to be accounted for.

% ===================================================================================================================================
\section{Charities, what are they good for?}
% ===================================================================================================================================
This section explores Charities and Non-Profit Organisations (NPOs); how they are defined, and what role they play in society. This is done for two reasons: firstly, to explore the ecosystems, landscapes, and settings within which these organisations operate so that the research is effective; and secondly, to ground the work's relevance as playing a part in the everyday activities of the world.

%%::::::::::::::::::::::::::::::::::::::::::::::::::::::::::::::::::::::::::::::::::::::::::::::::::::::::
\subsection{What is a Charity?}
%%::::::::::::::::::::::::::::::::::::::::::::::::::::::::::::::::::::::::::::::::::::::::::::::::::::::::

Defining what constitutes a charity can be problematic because it is a specific form of organisation that belongs to an entire sector or family of organisations which have historically resisted definition \cite{salamon_search_1992-1, morris_defining_2000}. This is largely due to the sheer diversity of both the organisations themselves as well as the legal and social frameworks in which they operate \cite{salamon_search_1992-1}. Even choosing which term to use is problematic not only because any given term can emphasise particular traits of organisations or exclude some organisations entirely, but choosing what term to use will give any discussion a particular national flavour. For example, the term 'Charity Sector' is often used in the UK whereas framing this discussion using the term of 'Non-Profit Organisations' (NPOs) makes it feel distinctly relevant to the USA \cite{frumkin_being_2009}. However, as noted, these organisations all share a genealogy, which means utilising literature that in turn uses a variety of terms to describe this group of organisations. A working definition of 'a charity' will be outlined at the end of this section.

Charities are a form of Non-Profit Organisation (NPO) which operate within what is often known as the "Third Sector" of the economy; emphasising their separation for public or state-owned operations as well as private for-profit enterprise \cite{salamon_search_1992-1}. The term 'Third Sector', however, is often used interchangeably with others such as "Voluntary Sector", "Independent Sector", "Charitable Sector", or many others. Salamon and Anheier claim that this abundance of definitions often poses a problem, as each term emphasises a particular characteristic of these organisations whilst downplaying others -- which can be misleading when attempting to describe them \cite{salamon_search_1992-1}. An example of this would be how the term "Voluntary Sector" emphasises the contributions of volunteers in the operation of the organisations, at the expense of organisations or activities that are performed by paid employees. Frumkin prefers the term "Non-Profit and Voluntary Sector" for this reason \cite{frumkin_being_2009}.

This diversity of organisations within the "Third Sector" means that a general definition is difficult to generate, however Salamon and Anheier go some way to provide one based off of the structural or operational characteristics of the organisations; which would therefore allow their definition to cater for the sector's diversity of legal structures, funding mechanisms, and function. Their definition identifies five base characteristics common to the organisations (they use the term NPOs). These organisations are: "Formal", having been constituted or institutionalised legally to some extent; "Private", meaning they are institutionally separate from government; "Non-Profit distributing", where any profits generated by activities are reinvested directly into the 'basic mission of the agency' instead of being distributed to owners or directors; "Self-governing", with their own internal protocols or procedures as opposed to being controlled directly by external entities; and "Voluntary", where the organisation's activities or management involves a meaningful degree of voluntary participation \cite{salamon_search_1992-1}.

Frumkin gives three characteristics of these organisations which align with Salamon and Anheier's framework. The organisations: do not coerce participation (ie they do not have a monopoly and interacting with them is optional); their profits are not given to stakeholders; and they lack clear lines of ownership and accountability \cite{frumkin_being_2009}. These definitions are not without issue, as they notably exclude various quasi-commercial entities such as those found in the UK -- ie Building Societies and Cooperatives.

It is the exclusion of such entities that presents an issue for achieving a working definition. Lohmann calls for a more expansive view of the "Nonprofit Organisation" since definitions often account only for those legally bound by particular legislation and that if academics work only within these confines then they are limited in their attention \cite{lohmann_charity_2007}. Lohmann also takes issue with the term "Third Sector" as it often is not presented in context of what it is a sector of. Lohmann argues that the organisations generally included in definitions of the "Third Sector" are actually simply a part of a broader grouping termed the "Social Economy" which would include NPOs and Charities but also others such as cooperatives and member organisations \cite{lohmann_charity_2007}. Moualert and Ailenei elaborate that the term "Social Economy" is tied with notions of economic redistribution and reciprocity, and argue that a "one-for-all" definition is not useful to produce, as the organisations within the Social Economy are driven by local contexts \cite{moulaert_social_2005}. They put forward that the Social Economy as a practice, as well as the institutions that make it up, are linked to periods of crisis -- and that the Social Economy is a method to respond to the alienation and dissatisfaction of people's needs by the For-Profit and State sectors at any given time \cite{moulaert_social_2005}. Monzon and Chaves go into detail about defining the characteristics of organisations that make up the Social Economy, largely echoing the definitions for the US-centric "NPOs" discussed earlier \cite{monzon_european_2008}. In addition to this is their elaboration that "[the organisations] pursue an activity in its own right, to meet the needs of persons, households, or families … [They] are said to be organisations of people, not of capital … They work with capital and other non-monetary resources but not for capital." This indicates that the unifying characteristic of these organisations is their concern for people, and begins to define them based on what they are rather than the via negativa of "Third Sector" \cite{monzon_european_2008}.

In the UK, the term 'Charity' is protected and has a specific definition enshrined in law. According to the Charities Act 2011, a Charity is an organisation that is "established for charitable purposes only", where the Act then later defines a list of charitable purposes to ensure that the organisation is acting for the public benefit \cite{noauthor_charities_2011}. These cover a wide variety of purposes and such as "the prevention or relief of poverty" and "the advancement of citizenship or community development''. Whilst this mirrors the Monzon and Chaves assertion that organisations pursue activity to "meet the needs of persons…", it is the opinion of this thesis that enshrinement in law is not necessary treat an organisation as a charity for the purposes of research. This is so that any outcomes of the research can be applied to intenational contexts -- where different legal definitions of the word "Charity" may exist. To that end, our definition of a charity going forward takes the common threads discussed in this section that Charities are: non-for-profit organisations that are legally distinct from government; are set up towards a charitable purpose (regardless of whether that purpose is enshrined in law); and that a citizen's interactions with the organisation are voluntary.


%%::::::::::::::::::::::::::::::::::::::::::::::::::::::::::::::::::::::::::::::::::::::::::::::::::::::::
\subsection{Why are Charities Important?}
%%::::::::::::::::::::::::::::::::::::::::::::::::::::::::::::::::::::::::::::::::::::::::::::::::::::::::
Charities are seemingly inherently valued by most individuals in civil society. The social motivations behind Charities and the wide variety of activities in which they involve themselves, as well as the manner of their involvement often means that the health of the Third Sector and Social Economy (TSSE) are often used as barometers for the health of civic society \cite{moulaert_social_2005}. This thesis therefore seeks to explore the importance of Charities and the TSSE to society in order to understand better the world in which they operate.

Hannsman writes on the role of Charities and the TSSE that they often emerge from a "contract failure" of the market to police the producers of services, and that it is very rare to find Charities operating in industrial sectors \cite{hansmann_role_1980}. According to Hansmann, economic theory dictates that the failure is in accordance with consumers (as a group rather than individuals) to do one of the following: accurately compare providers; reach agreement  as to the price and quality of services to be exchanged; and to assess the compliance of the organisation to their part of the deal, obtaining redress if the organisation is seen to have not complied. Charities and TSSE Organisations emerge, therefore, when this process has failed to regulate For-Profit actors in any given economic activity: "The reason is simply that contributors [to a for-profit business] would  have little or no assurance that their payments … were actually needed to pay for the service they received" \cite{hansmann_role_1980}. As noted, it is uncommon to find Charities and TSSEs operating in industrial sectors, and as such the services offered by these organisations can often be those that involve a separation between the purchaser of a service and the eventual recipient; e.g. the purchase and transport of food aid overseas. The inability of Charities to distribute profits to shareholders thus removes the incentive and power of organisations to reduce direct spend on the service; reassuring the purchaser that their money is not for the direct profit of shareholders \cite{hansmann_role_1980}.

Salamon writes that Charities and TSSEs "deliver human services, promote grass-roots economic development, prevent environmental degradation, protect civil rights, and pursue a thousand other objectives formerly unattended or left to the state" \cite{salamon_search_1992-1}. This insight reinforces Hansmann's view that the activities of these organisations are concerned primarily with provision of services unattended to by For-Profit sectors. Salamon's statement also implies the presence of State actors in a given activity and that State-provided services would mean that there is no requirement for a Charity actor if the needs of the people were being met. Frumkin argues that a core part of the TSSE is that it is responsive to demand; specifically the demands of a public who have unmet needs \cite{frumkin_being_2009}. Not only does Frumkin's argument add weight to both Hannsman and Salamon's admonition that the For-Profit sector is either unconcerned or untrusted with particular activities, but also that the State is either an absentee actor or that the service provided is unsatisfactory in meeting the needs of the public.

The nature and scope of activities in which Charities and the TSSE are involved are incredibly diverse. Salamon and Anheir outline a classification system, the International Classification of Nonprofit Organisations (ICNPO), that divides and classifies organisations into 12 groups based on economic activity, with an additional 24 sub-groups \cite{salamon_search_1992}. Whilst this classification system generally only provides high-level descriptors of organisations, lacking detail on the nature of how activities are performed pragmatically on-the-ground, they offer a starting point from which to begin to understand the far-reaching and diverse nature of the TSSE's activities. Examples range from "Nursing Homes" and "Mental Health and Crisis intervention", to "Housing" and "Culture and Arts".

The activities undertaken by Charities and the TSSE are also important to society because they are generally understood to produce and sustain Social Capital \cite{king_social_2004, wang_social_2008, swanson_strategic_2013}. Generally, Social Capital is the term used to refer resources and access to those resources as permitted by one's social network \cite{field_social_2003}. Putnam defined Social Capital as "features of social organisation, such as trust, norms, and networks, that can improve the efficiency of society by facilitating coordinated actions" \bollocks{putnam citation needed}. The 'resources' in Social Capital may be physical resources (ie tools) or more intangible types of resource such as possessing a skill or qualities that are valuable to society.

The amount of Social Capital an individual (or group) possess can have substantial effects on their day-to-day lives in a variety of areas. Field discusses how an increased amount of Social Capital has effects on personal health and happiness, as well as the educational prospects of one's children, and the amount of "anti-social" behaviour present in their communities \cite{field_social_2003}. Conversely, low amounts of Social Capital within communities can manifest as poor socio-economic conditions such as higher crime rates and low employment. Field writes about two flavours of Social Capital: 'bonding capital', which strengthens bonds between sociologially similar groups such as close friends and family; and 'bridging capital' which connects members to existing networks originally distinct to their own \cite{field_social_2003}. Bourdieu discusses how the bonding capital can be a means to denote or sustain privilege in society (ie the Old Boys' Clubs), and Putnam similarly states that whilst bonding capital can get one by, bridging capital is required to 'get ahead' \bollocks{Bourdieu and Putnam citations needed}.

With this in mind, it becomes easier to understand how the activities undertaken by charities and the TSSE are linked to the health of society. As discussed, their activities are generally grassroots in nature and as such can involve producing bonding capital between actors who are their beneficiaries in addition to providing opportunities for developing bridging capital that people would otherwise not be presented with. It is also worth noting that Field discusses that there requires an investment in more than just network building in order for society to benefit from Social Capital -- the individuals who form the network must also learn skills in order to benefit each other \cite{field_social_2003}. This is also an activity that is generally attended to by charities and the TSSE; organisations within this sector often concern themselves with benefiting others in the form of 'skills development' of either specialist forms or of a more generalised and transferrable nature that were denied to them because of their existing sociological standing \bollocks{Citation needed}.

%%::::::::::::::::::::::::::::::::::::::::::::::::::::::::::::::::::::::::::::::::::::::::::::::::::::::::
\subsection{What are the challenges of Charities today?}
%%::::::::::::::::::::::::::::::::::::::::::::::::::::::::::::::::::::::::::::::::::::::::::::::::::::::::
Like any organisation, charities experience a set of pressures dependent on their circumstances, with the heterogeneity of the sector meaning that each individual organisation will be subject to unique pressures. Generally, however, it is understood that there are a range of pressures that operate on Charities and the TSSE across the board.

As of writing, in the UK we have experienced nearly a decade of austerity politics which has resulted in significant reduction of funding to national services as well as Local Government Organisations (known as Councils) \bollocks{citation needed}. The result of this is that Charities and the TSSE are having to supply people with the services that they require either independently providing services that were once provided by the Councils, or working as a contracted official supplier of a service once provided "in-house" \bollocks{citation needed}. At the same time, the change in national leadership associated with the austerity agenda has lead to uncertainty in UK charities securing adequate funding to meet their needs as government grants are reduced or disappear entirely \bollocks{citation needed}.

In response to this shifting environment, many charities and TSSE organisations are switching their operational model to that of a Social Enterprise (SE) or Social Entrepreneurship in general \cite{borzaga_emergence_2004}. SEs are, yet again, a diverse set of organisations --- but one that specifically combines business-like elements, activities and structures from the For-Profit sector and applies them to activities that are intended for social betterment and benefit to society; much like traditional charities and TSSE organisations \cite{defourny_social_2008, doherty_diverse_2006}. Dart broadly describes Social Enterprise as "significantly influenced by business thinking and by a primary focus on results and outcomes for client groups and communities" \cite{dart_legitimacy_2004}, and Dees states that Social Enterprise combines the passion of a social mission with an image of business-like discipline \cite{dees_meaning_1998}. In practice, this often includes activities and practices that include revenue-source diversification, fee-for-service programs, and partnerships with the private sector \bollocks{citation}. Defourny and Nyssens describe the rise of SEs across the world, and that the UK has used the SE 'brand' within policy documents for years, and give as part of the working definition that the organisations profits are "principally reinvested for [a social mission] in the business or community, rather than being driven by the need to maximise profit for shareholders and owners" \cite{defourny_social_2008}. This definition shares similarities to that for charities and TSSEs discussed earlier in the review -- however distinctly does not include the requirement that profits cannot be distributed to shareholders, only that they principally are used primarily towards an organisation's social mission.

SEs often cited as a solution to the issues being experienced by charities and the TSSE, but it is not without criticism. Eikenberry writes that organisations adopting a social enterprise model actually poses a threat to civil society \cite{eikenberry_marketization_2004}. She argues that change in model leads to a focus on the bottom line and overhead expenditures, and exposes them to market forces that they would otherwise be sheltered from. Aside from the effects on the organisation itself, this exposure means organisations can often adopt "market values" and "entrepreneurial attitudes" which means that the change in their operational model is detrimental to society \cite{eikenberry_marketization_2004}. Dart elaborates that SEs differ from traditional NPOs as they generally blur boundaries between nonprofit and for-profit activities, and even enact "hybrid" activities \cite{dart_legitimacy_2004}. This could include activities such as engaging with marketing contracts as opposed to accepting donations, as well as behaviour such as cutting services that are not deemed to be cost-effective. Whilst there are large implications for organisations accepting funding from for-profit industries, a major implication of changing operational model is that the shift of effort from effective service delivery to financial strategy impacts negatively on the Social Capital that is generated \cite{eikenberry_marketization_2004}. This is through less emphasis on building relationships with stakeholders (previously an essential survival strategy) as service users become framed as consumers, and through market pressures diverting resources towards skills such as project management and away from activities that build Social Capital. Doherty et al. echo this in their description of Social Enterprises, distinguishing them from traditional models of charities and TSSE organisation by stating that the latter are "more likely to remain dependent on gifts and grants rather than developing true paying customers" \cite{doherty_diverse_2006}. Eikenberry's concerns are manifested here, as the service user or "beneficiary" of an organisation becomes reframed as a "customer" due to the influence of market forces.

Social Capital plays a significant role in the success of a charity or TSSE organisation. As actors within social networks themselves, these organisations need to make use of Social Capital in addition to their pivotal role in producing and sustaining it for others. King writes that charities and TSSEs were formed using Social Capital and part of their role is to "sustain and broaden" it in order to provide opportunities and make the mundane operation of an organisation smoother. She writes that those in leadership positions within an organisation draw upon techniques such as networking and skills development in order to allow the organisation to perform its work and meet its goals -- calling charities and TSSEs (she uses the term nonprofits) "the epitome of Social Capital in action" as the organisations can not only utilise but spread their Social Capital to others. Swanson shares these sentiments and explicates that strategic engagement of an organisation's Social Capital should be a central tenant in its management and leadership, Fredette and Bradshaw echo this and discuss how bonding capital established between those in leadership roles allows them to collectively mobilise through the sharing of information and the building of trust.

Trust is inextricably tied to Social Capital, as Field discusses that a network with high trust levels operates more efficiently than one with comparably lower levels of trust. This means that in order for a charity and TSSE actor to achieve its goals more effectively, it must be trusted. Whilst it is important to note that there is some disagreement as to the exact nature of Trust within Social Capital ie whether Trust is a product or instigator of Social Capital; it remains that high levels of Trust allows an organisation to operate more effectively, and continue the cycle of production and sustenance of Social Capital for their stakeholders. Schneier writes on Trust that it is essential for society at large to function (e.g. we trust in our currency, we trust in our qualifications etc.), although on-the-ground Trust plays a key role in accessing resources in the social network, since a transaction between two trusting actors is less expensive (both in terms of emotional labour and financial capital) to faciliate than a similar transaction between two actors lacking trust. Trust, therefore, is an important factor for charities and TSSEs in the performance of their work as lack of Trust will impede an organisation as much as high Trust will aid them.


\subsection{Summary}
It can be said, then, that since charities and TSSEs perform work that is important to society and needs to be performed, and that since high levels of Trust allows them to operate more effectively; that it is important to society that we trust our charities and TSSE organisations to perform the work that they do. However, recent media coverage (at least in the UK) has often portrayed Charities and TSSE organisations as being irresponsible with funding, ineffective in achieving the outcomes they purport to desire, and in some cases unaccountable for their actions. This review now turns to examining the concepts and mechanisms to which charities and the TSSE can often be subject to related to their transparency and accountability.

%% ===================================================================================================================================
\section{Why is Transparency important to NPOs?}
%% ===================================================================================================================================
%% What do we mean by transparency?
%% What are an NPO's responsibilities
%% What are their reporting requirements? (To funders, to the public?)
%% What does it mean to be transparent as an organisation?
%% What are the measures taken to ensure NPOs are transparent?
%% How does transparency relate to an organisation's accountability?

\bollocks{This introduction need rewriting to tie it more directly with why it's important we look at it if we're looking at charities}

This section explores Transparency and Accountability in the context of Charities and the TSSE. This is done so that we may understand the mechanisms by which these organisations may become more trustworthy to their stakeholders, facilitating not only their daily operation (as discussed above) but in doing so; continue producing value for society at large.

In understanding the roles that organisational Transparency and Accountability may play in this, we situate the research as operating within these spaces in order to provide a foundational understanding from which to begin working.

Transparency and Accountability are seen increasingly desirable in governments and organisations \cite{hood_accountability_2010, oliver_what_2004, heald_fiscal_2003}. Oliver states that Transparency has \quoteit{moved over the last several hundred years from an intellectual ideal to center stage in a drama being played out across the globe in many forms and functions} \cite{oliver_what_2004}.  Corr\^ea et al. say Transparency and Open Government is \quoteit{synonymous with efficient and collaborative government} \cite{correa_really_2014}, and Steele goes as far to say \quoteit{Transparency is the new `app' that launches civilization 2.0} \cite{steele_open-source_2012}.
%
Non-Profit Organisations (NPOs) in the UK are held to stringent Transparency standards by an organisation known as the Charity Commission, which is responsible for registering and regulating charities in England and Wales \quoteit{to ensure that the public can support charities with confidence} \cite{hm_government_charity_????}. The development of trust is foundational in the relationship between an organisation and those invested in its activities or performance, known as stakeholders, which is compounded by the notion that a stakeholder in an NPO might not be in direct receipt of its services \cite{macmillan_relationship_2005, krashinsky_stakeholder_1997}. Beyond this, accountability is seen as a way of building legitimacy as an organisation \cite{anheier_accountability_2009}. Watchdog organisations such as the Charity Commission and others therefore play an important role in developing stakeholder relationships with NPOs through Transparency measures, making them accountable to those invested in them. Oliver writes that NPO expenditure is often the \quoteit{most emotional}, and a person's decision to invest in a charity will be down to how comfortable and confident they are in its operation \cite{oliver_what_2004}.


Hood writes that \quoteit{Transparency is more often preached than practised [and] more often invoked than defined} \cite{hood_transparency_2006-1}. This section considers various definitions of Transparency in relation to the UK Charity Commission, NPOs, and the measures that are taken to make them accountable to stakeholders. It also inspects Transparency's synonymity with Accountability.

%% ::::::::::::::::::::::::::::::::::::::::::::::::::::::::
\subsection{What is Transparency?}
%% ::::::::::::::::::::::::::::::::::::::::::::::::::::::::



Transparency and Accountability are seen as increasingly desirable traits in governments and organisations, and have moved over the last century from intellectual ideas in the wings to playing a central role across the globe. Correa et al say of Transparency and Open Government that it is "synonymous with efficient and collaborative government", and Steele writes that "Transparency is the new app that launches civilization 2.0". However, the terms are still ambiguous. Hood writes that Transparency is "more often preached than practiced [and] more often invoked than defined". This section therefore aims to explore various definitions of 'Transparency'.

Historically, Transparency was inherent in the actions and interactions of everyday society since, in traditional societies, the density of social networks made one's actions highly visible. Meijer contrasts this with modern societies where "people do not know each other -- many people in cities do not even know their neighbours" and argues that societies which operate at a larger scale suffer a decline in social control, which calls for new forms of Transparency that match the scale of the society. The term Transparency has been a watch-word for governance since the late 20th century, yet its roots stretch back much further. Hood identifies three `strains' of pre-20th-century thought that are at least partial predecessors to Transparency's modern doctrine: rule-governed administration; candid and open social communication; and ways of making organisation and society `knowable'.

The first of these "strains" of thought, Rule-governed administration, is the idea that government should operate in accordance to fixed and predictable rules and, and Hood calls it the "one of the oldest ideas in political thought". This notion may be summarised effectively with the platitude of "a government of laws and not men", where the laws are stable and governing is thus not subject to the discretionary attitudes of individuals. The second strain, Candid and Open social communication, had its early proponents liken Transparency to one's "natural state", and it saw an implementation in the `town meeting' method of governance where members of the town would deliberate in the presence of one another -- making all deals transparent and ensuring all parties were mutually accountable. The third of form of proto-Transparency doctrines, is the notion the social world can be made 'knowable' through methods or techniques that act as counterparts to studying natural or physical phenomena. Hood describes an 18th-century "police science" which exposed the public to view through the introduction of street lighting or open spaces, as well as the publication of information (all of which designed to help prevent crime).

When viewed in this historical context, from these different perspectives, it can be said that Transparency is inherently concerned with information; access to it, and effective use of it. Oliver describes Transparency as having three key components: something (or someone) to be observed; someone to observe it; and the means supporting such an observation. Heald discusses how these first two components can manifest in modern Transparencies with a property of directionality; a direction being an indicator of who is visible to whom. Heald conceptualises four directions of Transparency that exist across two axis: Upwards and Downwards; Inwards and Outwards.

The 'vertical' axis (Upwards and Downwards) refers to position in a given hierarchy, such as found within an organisation or a nation. An Upwards Transparency would indicate that those higher in a hierarchy can observe the conduct, behaviour or actions of those below them, whilst Downwards Transparency would mean that those higher on the ladder are made observable by those below them. The 'horizontal' axis (Inwards and Outwards) refers to relative position to an organisation and whether one can observe or be observed by it. An Inwards Transparency would mean that those external to an organisation may see into it, and conversely an Outwards Transparency would denote situations wherein an organisation may peer 'out' and monitor its habitat or other actors.

These forms of Transparency can (and often do) coexist simultaneously in a given situation, and any combination. Further to this, operation of a direction across one axis does not preclude the existence of its counterpart. For example, there may be a situation which can be described as possessing both Upwards and Downwards Transparencies. When this occurs Heald describes that axis as having "symmetry". Real world examples can be analysed in this manner; a government's surveillance of its citizens (or a private company's surveillance of its workers) can be described as a combination of Downwards and Outwards Transparencies. The inverse of this situation, Upwards and Inwards, has also been encapsulated with the term 'Sousveillance' -- a term coined by Mann meaning "to watch from below". Mann gives two possible interpretations of the term Sousveillance. The first, as discussed, is an inversion of Surveillance formed by Upwards and Inwards Transparencies allowing citizens to capture abuses of power by those in positions of authority such e.g. by police officers at street level. The second interpretation of Sousveillance specifically refers to the relative positions of cameras in physical space such as the proliferation of cameras attached to modern smartphones. It can be argued that this has enabled the first form of Sousveillance, and that abuses of power that have always occurred and that they are only now being witnessed en masse. However, with the advent of digital monitoring endorsed by governments and corporate bodies, and the possibilities of these actors utilising citizenry's smartphones -- this means that Sousveillance can also have implications for the Downwards and Outwards forms Transparency discussed in the context of traditional surveillance. In addition to this, Ganascia also discusses how an increased desire for access to public information has lead to aspirations for "total transparency", which ion the US has resulted in government endorsement of data sharing (Sometimes called 'Open Government' or 'Government 2.0'). There have been similar moves in the UK (e.g. data.gov.uk) which are designed to improve the delivery and transparency of public services.

The sharing of data, however, does not constitute the entirety of Transparency. As discussed, the historical context of Transparency appears concerned with two aspects of information; access to it is indeed one of these, however the effective use of such data is also an important factor. Schauer writes of Transparency that it cannot be simply equated with knowledge, and at best facilitates it. For information or processes to be Transparent he defines the criteria of being "open and available for scrutiny", but this definition notably lacks an explanation how groups or individuals may make use of information, and the cost for them to access it. Hood also acknowledges this as a tension between the historical "Town Hall" forms of Transparency, related to Candid and Open Social Communication, and its distant cousin concerned with accounting and book-keeping.

It is this concern with accounting and book-keeping that is most often associated with Transparency in common parlance. In this context there are also distinct flavours of Transparency that must be acknowledged. In government and business, Transparency has taken the form of releasing information concerned with accounts and expenditure on a regular or semi-regular basis. Oliver discusses how this is an older form of Transparency and almost purely reactive -- often in response to a scandal. Further to this, Oliver writes that the Old Transparency is giving way to what he calls the 'New Transparency', which is more proactive and the taking on a stance of "active disclosure". Similar to Oliver, Schauer provides a discussion on the dualistic nature of Transparency divided across the same lines of passiveness vs activity -- to the point where he names the twin forms of Transparency "Passive Transparency" and "Active Transparency". From this point, Oliver and Schauer's discussions converge along similar lines. Old and Passive Transparency is concerned only with information being made available "for others to see if they so choose, or perhaps think to look, or have the time, means and skills to look"; which resembles the discussion that definitions of Transparency often don't consider how stakeholders may access or understand information about an actor. New and Active Transparency is not only demanding, but concerned with information's interpretation and access and should be thought of as of communication concerned with the organisation's responsibilities. Heald discusses a very similar division of Transparency which he calls "Nominal Transparency" vs "Effective Transparency". The term "Nominal Transparency" describes something similar to the Old and Passive Transparencies outlined by both Oliver and Schauer, but more ominous. Heald says that whilst Transparency of any given organisation may increase on an index, there is a divergence with Effective Transparency to the point where it is Transparency only in name -- creating an illusion of Transparency. For Transparency to be effective, Heald writes that there must be receptors capable of receiving, processing, and utilising the information.

Heald's dichotomy between Nominal and Effective Transparency sits alongside two other similar dichotomies that he describes as being important factors to discussion of the term: Real-Time vs Retrospective Transparencies; and Event vs Process Transparencies. The first dichotomy deals with the variable of time in the availability of information as a Real-Time Transparency would take a form of continuous surveillance such as that enabled by modern technologies such as CCTV or (a little less odious) open data apis which are continuously fresh. Retrospective Transparency describes a reporting cycle during which an organisation operates and then prepares an account of activity. The second pairing of Event and Process Transparencies concerns the subject of the Transparency. Event Transparencies describe objects or states that that are more easily measurable than their counterparts, Processes, which are more likely to be described in attempt to be Transparent rather than reported on. Events and Processes are inherently linked, as it requires a Process to turn one Event into another form of Event, such as an Input into an Output via a Transformation process. A concrete example of this would be financial input being transformed via action or spending into an output, and then later linked into an outcome for reporting. Of these Events, Inputs are the easiest to measure and can be measured directly. Outputs can also be measured although such measurements are effectively proxies related to activities undertaken, and linking these to outcomes can be a difficult or impossible task.


%% :::::::::::::::::::::::::::::::::::::::::::::::::::::::::::::::::::::::::::::::::::::::::::::::::::::
\subsection{NPO Reporting Requirements and Responsibilities}
%% :::::::::::::::::::::::::::::::::::::::::::::::::::::::::::::::::::::::::::::::::::::::::::::::::::::
\bollocks{Discuss the principal-agent theory of accountability in npos}
%% ::::::::::::::::::::::::::::::::::::::::::::::::::::::::::::::::::::::::::
\subsection{Transparency and Accountability}
%% ::::::::::::::::::::::::::::::::::::::::::::::::::::::::::::::::::::::::::

% Introduce idea that transparency and accountability might not be linked
% Fox says that, aside from the muddle of transparency, there is also a muddle around accountability (define accountability)
% State question about different types of transparency generating different types of accountability
% Define the different types of transparency and accountability, make sure you discuss the focus on transparency goals (individual vs institutional)
% Make the point aht there is a difference between the official data, and the reliable RELEVANT information, leads to the fuzzy transparency , which means that an investment is 				required to transform data into transparent information.
% Introduce answerability, and then state that answerability is predicated on the ability to produce answers

The imposition of transparency measures is generally seen as tantamount to ensuring accountability of institutions, organisations, or individuals in power; and often the terms are used interchangeably \cite{fox_uncertain_2007, hood_accountability_2010}. The two terms, however, are separate and have their own (if somewhat malleable) definitions \cite{fox_uncertain_2007}.
%
Fox discusses accountability in terms of \quoteit{the capacity or right to demand answers} or the \quoteit{capacity to sanction}, whereas transparency concerns itself with the public's right and ability to access information; and whilst common wisdom dictates that transparency generates accountability, this assumption is challenged when held to scrutiny \cite{fox_uncertain_2007}. Fox's analysis of transparency divides it into two categories -- \textit{clear transparency} and \textit{opaque} or \textit{fuzzy transparency} -- which closely resemble Schauer and Oliver's definitions of the \textit{Active} or \textit{New Transparency} and the \textit{Passive} or \textit{Old Transparency} \cite{schauer_transparency_2011, oliver_what_2004, fox_uncertain_2007}. Fox argues the importance of this distinction lies in the fact that as Transparency becomes an increasingly desirable term, opponents will express their dissent through provision of \textit{fuzzy transparency}. This is data which lacks information that can reveal organisational behaviour and thus cannot be used to generate accountability \cite{fox_uncertain_2007}.
%
The \textit{Clear Transparency} alluded to by Fox is defined as \quoteit{information-access policies [and] programmes that reveal reliable information about institutional performance, specifying officials' responsibilities [and] where public funds go}. Importantly, though \textit{Clear Transparency} is concerned with organisational behaviour, it is not sufficient to generate accountability -- which requires the intervention of other actors \cite{fox_uncertain_2007}. Accountability is also explained by Fox as either \textit{Soft Accountability} (the ability to demand answers) and \textit{Hard Accountability} (the ability to issue sanctions). Fox stipulates that appropriate levels of \textit{Clear Transparency} gives the public the ability to perceive problems, and to demand answers -- which is a form of \textit{Soft Accountability} known as \textit{answerability} \cite{fox_uncertain_2007}. Further forms of accountability are founded on the ability to not only reveal existing data, but to investigate and produce information about organisational behaviour \cite{fox_uncertain_2007}.
%
Anheir and Hawkes reflect Fox's sentiment in their discussion of Accountability, where they describe Accountability as a \quoteit{multi-dimensional concept that needs unpacking before becoming a useful policy concept and management tool}, and note that in the case of trans-national organisations; Accountability itself is a problem, and not simply a solution \cite{anheier_accountability_2009}. This discussion, whilst focusing on the difficulty of regulating accountability across national borders, has insight into the ways that transparency mechanisms may not be adequate for generating true accountability in NPOs. They highlight how it is often media companies that reveal `unethical behaviour' to the public, rather than formal auditing bodies -- an example from the UK would be how the NPO \textit{Kid's Company} experienced negative media coverage over their closure relating to alleged misuse of funds \cite{elgot_kids_2015, anheier_accountability_2009}. Anheir and Hawkes also draw on Koppel's `Five Dimensions of Accountability' framework -- which imbues Accountability with a five-part typology: \textit{transparency}; \textit{liability}; \textit{controllability}; \textit{responsibility}; and \textit{responsiveness} \cite{anheier_accountability_2009, koppell_pathologies_2005}.
%
Koppel avoids trying to produce a definitive definition of Accountability, stating \quoteit{[to layer] every imagined meaning of accountability into a single definition would render the concept meaningless}, and the five-dimensional typology is instead designed to facilitate discussion of the term \cite{koppell_pathologies_2005}. Transparency features prominently in the typology, with Koppell referring to it as one of the \quoteit{foundations, supporting notions that underpin accountability in all of its manifestations} alongside liability \cite{koppell_pathologies_2005}. Liability, according to Koppell, is the attachment of consequences to performance and culpability to Transparency -- punishing organisations or individuals for failure, and rewarding them for successes \cite{koppell_pathologies_2005}. In this, the `foundational' dimensions of Koppell's typology are aligned with Fox's definitions of Accountability which covers the capacity of demanding answers and to sanction, with Transparency as the ability to access the information in the first place \cite{koppell_pathologies_2005, fox_uncertain_2007}.
%
The remaining three dimensions of Koppell's typology: \textit{controllability}; \textit{responsibility}; and \textit{responsiveness} are all built upon the foundations of Transparency and Liability. \textit{Controllability} is a form of accountability where if \quoteit{X can induce the behaviour of Y [then] X controls Y [and] Y is accountable to X} \cite{koppell_pathologies_2005}. Koppell notes that \textit{Controllability} can be difficult in organisations that have multiple stakeholders to whom the organisation is supposed to be controlled by \citep{koppell_pathologies_2005}. \textit{Responsibility} denotes the constraint of behaviour through laws, rules, or norms such as legal frameworks or professional standards of conduct. \textit{Responsiveness} in the typology describes the attention of an organisation to the needs of its clients, as opposed to the following of hierarchical orders \cite{koppell_pathologies_2005}.
%
\bollocks{MAD stuff here}.
%
\bollocks{if the charity commission is trying to ensure all aspects of accountability, is it subject to MAD?}
%
%% ===================================================================================================================================
\section{How can digital technologies support NPO Transparency and Accountability?}
%% ===================================================================================================================================


% General things -- Open Data and Human Data Interactions
Transparency and Accountability can be said to be ultimately concerned with the sharing of information and the creation of pathways or mechanisms that allows stakeholders to act in accordance to it. Meijer argues that \quoteit{modern transparencies are computer-mediated} \cite{meijer_understanding_2009}, and Oliver goes as far to posit that digital technologies have sparked a self-sustaining \textit{Information - Transparency Cycle} which is \quoteit{unstoppable} and that information is now a commodity which is cheap to collect, organise, analyse, and distribute; the result of which is a reaction to missing information and a return to the collection phase \cite{oliver_what_2004}. Similarly, Steele, in his \textit{Open Source Everything Manfiesto} reflects on the ways in which the Internet has enabled the public to overcome previous restrictions on access to information and states, in no unclear terms, \quoteit{This bodes well for humanity} \cite{steele_open-source_2012}.

Models which support openness and public access to information has use cases that support its effectiveness as a tool for certain forms of Transparencies and Accountabilities. The Free Software movement and, to a lesser extent, the Open Source movement\footnote{There are notable political and philosophical differences between the two. Free Software attempts to bring software tools back into the public Commons, whereas there are several notable critiques of Open Source as an implicit Capitalist movement attempting to subvert Free Software. If you're interested in this at all, I recommend reading Stallman's thoughts on the matter as a primer \cite{stallman_why_2002, stallman_why_2016}} have baked into their core the requirement for access to source code of software distributions; which is generally enforced through licensing code upon release in a particular way.

Camp frames the access to source of a given piece of software as a form of Transparency and Accountability \cite{camp_varieties_2006}, a sentiment shared by various advocacy groups promoting end-user interaction with the software tools over proprietary alternatives \cite{pfaff_open_1998, balter_6_2015}. Camp outlines how that human-readable source code (specifically, source code that is not deliberately obfuscated) can be 'audited' similar to to an 'open book' form of transparency. Camp then transposes these concepts into governance processes; where `open' code could be compared to digitised versions of an organisation's governance processes. With an `open' model, an organisation could be held to the same scrutiny as Open Source or Free Software source code \cite{camp_varieties_2006}.  

Free Software also demands a particular form of accountability through specific use-cases, namely that of programmers deriving work from it. Free Software is often released under 'restrictive' or `copyleft' licenses (e.g. \cite{free_software_foundation_gnu_2007}) which legally enforce that derivative software, or new software including Free-licensed code as a component, is also released holistically as under the same license and thus under the same terms -- enforcing access to the source. Stallman's original GNU manifesto outlines the reasons why his GNU system employs a form of viral licensing: \quoteit{'Control over the use of one's ideas' really constitutes control over people's lives; and it usually used to make their lives more difficult} \cite{stallman_gnu_1985}. In this, Stallman declares a unique form of accountability that can almost be seen as paradoxical -- one that explicitly controls the actions of a particular group (programmers) in order to dictate that they relinquish control over another group (end-users access to and subsequent use of software tools).

% Digital government stuff

% Linking work to be done
Broadly, there are several strands of research into digital technologies that support interacting with information and data in this way; primarily these can be encapsulated within the areas of Open Data, and Human-Data Interaction (HDI) although the inter-disciplinary nature of HCI as a whole means that the subject matter naturally intersects or otherwise touches upon other research within the field. This section explores the potential and implications for how data may be produced and appropriated by charities and their stakeholders through a discussion around Open Data, Open-Source Intelligence (OSINT) and Human-Data Interaction. It then turns to the pragmatic and investigates previous HCI research into previous examples pertaining to enabling interactions with finances




\subsection{Interacting with Data}

% Open Data and its uses
Open Data is \quoteit{data that anyone can access, use, or share} \cite{open_data_institute_what_nodate}. It consists of organised data that is, generally, structured and placed online so that it can be consumed for use. Open Data can be produced, shared, and used by many people in many different contexts (e.g. scientific data sets, or government collection of environmental data). Often, it is parsed or processed in some way by digital technologies, and multiple datasets may also be combined in order to produce a desired insight for the stakeholder(s) consuming the data.

% Human Data Interaction and its conception of the use of data
Human Data Interaction (HDI) is a coalescing field that is concerned with the social world of how people interact with data about themselves and others. Whilst it considers technical infrastructure surrounding data \cite{mcauley_dataware_2011}, HDI also brings data's role as a `Boundary Object' \cite{star_institutional_1989} to the fore and considers its role as a pervasive aspect of everyday life in terms of how to enable citizens to interact with this data in a more explicit fashion \cite{mortier_human-data_2014}.

A `Boundary Object' is anything which may be recognised across different social `worlds', yet may be appropriated and adapted by the needs of individuals and groups in a manner that pertains to their specific needs and context. Star and Griesemer describe Boundary Objects as \quoteit{both plastic enough to adapt to local needs ... yet robust enough to maintain a common identity across sites} \cite{star_institutional_1989} A good example of this is a receipt of purchase; it is inarguably a receipt, yet may be used by the bank to verify a purchase, by a store to prove that you own the items you've purchased, and proof that a transaction has occurred between your bank account and a store. Further to this, Crabtree and Mortier elaborate that Boundary Objects are ``inherently social'' and possess a ``processual character'' as part of the infrastructure of everyday life. To this end, they argue that Data is not so much an object in-and-of-itself but rather an object that is inherently embedded in human relationships \cite{crabtree_human_2015}.

Data's use as a Boundary Object is demonstrated effectively by the rise in personal informatics. In the Quantified Self movement, individuals collect and process various forms and sources of data about them as individuals, generally for the purposes of recording progress towards various goals \cite{swan_emerging_2009, swan_quantified_2013}. Contrasting the movement's general use of data as a pragmatic, goal-oriented object, Elsden et al demonstrate that data can be experienced by people in many ways and can serve different purposes such as providing playful ways to engage with each other and one's own data \cite{elsden_quantified_2014, elsden_metadating:_2016}. In particular, it is posited that data can offer an `alternative lens' that other media does not, allowing people to view or represent an event in a different ways than originally envisioned and one that can be combined with other more traditional forms of documenting; and in doing so indicates that the record is always unfinalised and is continuously open to reinterpretation \cite{elsden_designing_2017}.

Additionally, data as a Boundary Object has place within an organisational context, offering opportunities to use and present data in an exploratory context in people's shared worlds. In an academic context (most UK Universities possess charitable status and thus belong to the Third Sector), visualising research funding across the institution was found to act as a means of supporting members of staff in understanding the funding landscape of the organisation and in communicating narratives to the outside world around perceived successes. The system (and by extension, its data) was also found to support the review and contestation of data when multiple interpretations were available, and Elsden et al explicitly note its implications for organisational transparency; with the caveat that a major design question raised by the research is whether contextualisation should occur merely through the data, or in conversation around it \cite{elsden_resviz:_2016}.

As discussed, the presentation of information is not enough to engage in more modern forms of transparent practice; and the use of data (however nicely it may be visualised) is no exception to this and risks simply rehashing the older forms of transparency with faster production of data. Cornford et al write that the UK Government's agenda of producing Open Government Data (OGD) fails to address the questions of how information is to be interpreted for local contexts; mirroring the concerns of Elsden et al around how data should be contextualised. Cornford et al argue that a wealth of open and structured data merely provides a `view from nowhere' and that the true challenge lies in developing the interpretive communities that will utilise the information effectively \cite{cornford_local_2013}.

Some successes have been seen in the uses of OGD for engagement; the London Datastore \cite{noauthor_london_nodate} collects, organises, and distributes a variety of Open Data pertaining to the City of London in the UK, and has seen great successes in terms of use-cases where developers and organisations have used data for service provision and governance purposes \cite{coleman_lessons_2013}. In the US, journalists have begun utilising data to fuel their practice \cite{ramos_journalists_2013}, and it has been argued that the proliferation of OGD has directly contributed to a `habit of engagement' that in turn begins to develop a culture of civic participation through the use and responses to data about our civic world \cite{gordon_making_2013-1}. Black and Burstein conceptualise a movement towards the `Twenty-First Century Town Hall' \cite{black_local_2013} (possibly indicating a full circle movement to a more direct forms of Transparency discussed above) whereas Bloom postulates that OGD will form a new Commons resource \cite{bloom_towards_2013}.

% More generally, hint towards OSINT practices and bring in the Steele stuff a bit more heavily.
Sense making and engagement stemming from the use of Open Data draws upon the field known as `Open-Source Intelligence' (OSINT). Generally, OSINT concerns itself with the gathering of intelligence for problem solving from various public information sources \cite{bradbury_plain_2011, glassman_intelligence_2012}. This places it in contrast to other forms of intelligence-gathering which are generally performed using specialist or secret sources of information. Traditionally, this would look like utilising sources such as newspapers and public records but in the modern era sources such as Open Data and Social Media profiles may be used as viable sources of information to begin making steps towards solving intelligence problems \cite{bizer_emerging_2009}.

% Link OSINT to transparent practice; if questions can be asked and answered effectively, have we achieved a more substantial form of transparency?
The implications of an Open Data-fuelled OSINT for Transparency and Accountability are evident. As discussed, OSINT is concerned primarily with the application of information to solve issues or answer questions, so it stands to reason that an adequate data infrastructure would allow for (or even promote) engagement of stakeholders in OSINT for asking questions of charitable organisations (the `Habit of Engagement alluded to earlier \cite{gordon_making_2013-1}). At the very least, such an infrastructure would enable the organisations to produce effective and interactive responses to queries around the performance of their work and their spending. 

Steele discusses the power of OSINT at length. Coming from a position that ``the only unlimited resource in the world is the human brain'', Steele puts forward that the engendering of Transparency via the production of Open Data, and the resultant OSINT practices, would lead to a systematic practice of exchanging information openly. This would then allow societal actors and stakeholders to evaluate and respond to complex problems. Whilst Steele refers directly to engagement with Governmental processes, this could see use in the Third Sector as well; when a charity could present information about their work within the context of complexity. A pragmatic example that might be most interesting for charities is the example of a `True Cost' calculation --  wherein the True Cost of a white cotton T-Shirt is outlined in economic, societal, and environmental terms.

In the context of charities, Erete et al explore how NPOs use Open Data technologies to support their practice through practices resembling OSINT. Data is largely used to create a narrative and engage in a story-telling practice around particular goals which differ in context -- e.g. making grant applications, or internal management functions. Organisations are shown to combine multiple sources of data into a narrative, as well as being able to derive multiple narratives from a single data set. Further to this, they discuss how NPOs operate with limited resources and as such may benefit from services such as Data Portals to enable them to acquire data easily to produce these valuable narratives, and put forward that additional value is created via such portals when they act to build or strengthen relationships between those seeking to use the data and those possessing skills or knowledge around its analysis. From the perspective of Transparency and Accountability, it stands to reason that systems can be developed that allows charities (NPOs) to engage actively in the data collection process, and allow them to construct multiple, and varied narratives from personal data sets that can be used in similar contexts to those described by Erete et al -- ie supporting grant applications and internal management procedures, but also additional cases such as evidencing their work by retrieving and presenting information collected about it.

In summation, it appears that digital technologies surrounding Open Data and its use as a Boundary Object have strong implications for the support of Charities in terms of Transparency and Accountability. Use of data has in the past demonstrated a usefulness in civic and academic contexts, supporting processes that are integral to Transparency (review and contestation), as well as potentially acting as a vector to allowing stakeholders to explore the complexities orbiting particular topics such True Cost, and therefore spending. With this in mind the discussion now turns to previous HCI work in the area of interacting with finances.


% ::::::::::::::::::::::::::::::::::::::::::::::::::::::::::::::::::::::::::::::::::::::::::::::::::::::::::::::::::::::::::::::::::::::::::::::::::::
\subsection{How do digital technologies allow people to interact with finances?}%
% ::::::::::::::::::::::::::::::::::::::::::::::::::::::::::::::::::::::::::::::::::::::::::::::::::::::::::::::::::::::::::::::::::::::::::::::::::::
HCI research has previously concerned itself with investigating the ways in which can facilitate people's interactions around money. Work has largely been focused on small scale interactions such as those found at individual or family/small-group level. Examples include studies investigating how people manage personal finances in particular circumstances, as well as how people engaged with money on an experiential level. At larger scales, HCI has also taken into account the social world around financial transactions to theorise around the design of potential future payment systems; and alternative forms of capital such as cryptocurrencies and the surrounding Blockchain technologies have accountability baked into the infrastructure of the systems themselves.
%

Previous HCI work demonstrates that people interact with money and their personal finances in a number of ways. Kaye et al discuss how interaction can play out at an emotional level; as individuals may make decisions that do not appear 'rational' from a purely financial perspective but are instead driven by other factors such as personal history or experience with debt. A second facet is a form of management of 'pots' of money. In this context, money is not treated as a single entity but divided up along lines such as origin or intended use, and people use a variety of self-made or adaptable tools (both digital and analogue) in order to achieve this management; such as folders, notebooks, and spreadsheets. This practice of dividing money semantically is notably also shown in the work of Vines et al when studying techniques people use to manage a low income. Finally, dealing with the unknown or 'higher powers' is an important facet of people's relationships with money as they may lack important information held about them; such as their current Credit Scores (often used as a measure of financial health in the US), and they understand that their personal futures may contain events that they have not financially planned or accounted for. Vines et al go on to describe how the systems people implement can give them a 'confidence through awareness' which may act as a ballast that partly allays their fears.


At the community Scale, Ferreira et al explored the social world surrounding money, specifically a community currency known as the Bristol Pound. Their work discusses how exchanges of money using the currency shared aspects of a conversation, as the transactors would engage in social interactions that were unbounded by the settings roles such as 'shopkeeper' and 'customer', prompted by the technology use required to pay with the currency. Further to this, the use of the shared currency (and the technological systems supporting it) gave the transactors an indication of shared values and interests.

Other instances of digital technologies supporting group use of money is the use of `Crowdfunding' websites such as Kickstarter, or GoFundMe. In particular, the language and mechanisms of these sites share similarities with the process of donating to charities. These sites offer options wherein donators to a particular fund may have their donation returned if, for example, the total requested amount of donations has not been met. Beltran et al extend this concept further with their deployment of `Codo` which they describe as "Fundraising with Conditional Donations". In this deployment, Beltran et al describe how they developed a logical grammar which allows a donor on the system to more richly prescribe (or describe) the conditions of their donation, such as matching funds from other individuals or those within a defined group. Whilst this system does not proffer much in the way of exploring how organisations can report back on their expenditure, it presents the case that conditions may be put forward and codified as a means of providing a rudimentary accountability; as an organisation may need to attract the support of more than disparate groups in order to receive their donations. This opens up the possibility that a system may be developed with an 'accountability spin', where conditions are put upon funds by Funding bodies that request reimbursement under the event that conditions that they set out are not adequately met.

It is also important to note that my previous work has explicitly explored how Transparency and Accountability are, as of writing, poorly supported by digital technologies. This findings of this initial, exploratory, work highlights similarities between how individuals semantically divide money into 'pots' and how charities' finances are often restricted to particular use-cases due to how charity funding operates. This study also indicated that it may be appropriate to shift focus from financial transparency towards making organisation's 'visible'. This harks back to the historical roots of Transparency as a part-Science of making the social world knowable discussed earlier. The means to achieve this would be to produce a more qualitative form of accounting and supporting the interrogation of information collected by using standardised web technologies


% ===================================================================================================================================
\section{Opportunities}
% ===================================================================================================================================
% What is the financial and transparency work in NPOs?
% Lack in existing open data standards
% Lack in reporting processes
% Lack in the veracity of existing financial reporting for NPOS - i.e. transparency vs visibility
% Are these adequate?
% What can be done better?

% ===================================================================================================================================
\section{Summary}
\label{sec:related:conclusion}
% ===================================================================================================================================
